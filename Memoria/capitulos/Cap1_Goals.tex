%
% ---------------------------------------------------
%
% Proyecto de Final de Carrera:
% Author: Adriano dos Santos Moreira <alu0101436784@ull.edu.es>
% Chapter: Goals 
% File: Cap1_Goals.tex
%
% ----------------------------------------------------
%


\chapter{Goals} \label{chap:Goals}  
This document summarizes the research and development work carried out by the student in the achievement of his Final Degree Project (\textit{Trabajo de Fin de Grado}, TFG), which will conclude his studies for the degree \textit{Grado en Ingeniería Informática} at the \textit{Escuela Superior de Ingeniería y Tecnología} at the Universidad of La Laguna (ULL).

This project has the following main goals:

\begin{enumerate}
\item A first objective has been for the student to acquire basic knowledge on the topic of High Performance Computing. 
Achieving this objective requires an effort on the part of a student of the Degree in Computer Science at the ULL who has to familiarise himself with diverse concepts that are not studied in the syllabus of this degree.
Concepts such as performance, acceleration, latency, data parallelism, portability, accelerators, etc. have had to be studied on their own since they are not studied in the necessary depth in the syllabus, and others have also required effort on the part of the student to be understood with greater precision.

\item On the one hand, the aim is to broaden the knowledge of the parallel programming model using \textit{SYCL} \cite{URL::SYCL} and the development of applications for this scheme in the context of \textit{HPC} \cite{Assiroj:2018:High}.

\item Another objective of this work is to investigate and deepen in the techniques and technologies related to \textit{Data Parallel C++} \cite{Reinders:2023:Data} present today.

\item At the same time, the student is expected to acquire knowledge about the different implementations of the parallel programming model as well as the necessary tools and techniques to optimize its performance efficiently.

\item Along with the previous point, the student should also elaborate a comparative study between different parallel programming implementations, focusing on practical and execution related differences.

% \item It is also intended that the student improves their knowledge in the use of version control tools using \textit{Github} \cite{URL::Github} and technical text editing with \textit{LaTeX}  \cite{URL::LaTeX}.

\item Finally, after the corresponding research and information gathering, the student is expected to apply the acquired knowledge to develop some functional implementation that meets the proposed needs.
\end{enumerate}